\documentclass[12pt]{beamer}
\usepackage{../latex-sty/mypres}
\usepackage[utf8]{inputenc}
\usepackage[T2A]{fontenc}
\usepackage[russian]{babel}

\expandafter\def\expandafter\insertshorttitle\expandafter{%
  \insertshorttitle\hfill%
  \insertframenumber\,/\,\inserttotalframenumber}
\title[Выпуклые функции]{Методы оптимизации. \\
 Выпуклые функции.}
\author{Александр Катруца}
\institute{Московский физико-технический институт} 
\date{\today}

\begin{document}
\begin{frame}
\maketitle
\end{frame}

\begin{frame}{Определения функций}
\small
\begin{block}{Выпуклая функция}
Функция $f: X \subset \bbR^n \rightarrow \bbR$ называется выпуклой ({\color{blue}{строго выпуклой}}), если {\color{red}{$X$~--- выпуклое множество}} и для \\ 
$\forall \bx_1, \bx_2 \in X$ и $\alpha \in [0, 1] \; ({\color{blue}{\alpha \in (0, 1)}})$  выполнено:
\vspace{-4mm}
\[
f(\alpha \bx_1 + (1 - \alpha)\bx_2) \leq \; ({\color{blue}{<}}) \; \alpha f(\bx_1) + (1 - \alpha)f(\bx_2)
\]
\end{block}

\begin{block}{Вогнутая функция}
Функция $f$ вогнутая (строго вогнутая), если $-f$ выпуклая (строго выпуклая).
\end{block}

\begin{block}{Сильно выпуклая функция}
Функция $f: X \subset \bbR^n \rightarrow \bbR$ называется сильно  выпуклой с константой $m > 0$, если $X$~--- выпуклое множество и для $\forall \bx_1, \bx_2 \in X$ и $\alpha \in [0, 1]$  выполнено:
\vspace{-4mm}
\[
f(\alpha \bx_1 + (1 - \alpha)\bx_2) \leq \alpha f(\bx_1) + (1 - \alpha)f(\bx_2) - \frac{m}{2} \alpha (1 - \alpha) \| \bx_1 - \bx_2 \|_2^2
\]
\end{block}

\end{frame}

\begin{frame}{Определения множеств}
\begin{block}{Надграфик (эпиграф)}
Надграфиком функции $f$ называется множество $\text{epi}f = \{ (\bx, y) : \bx \in \bbR^n, \; y \in \bbR, \; y \geq f(\bx) \} \subset \bbR^{n+1}$
\end{block}

\begin{block}{Множество подуровней (множество Лебега)}
Множество подуровня функции $f$ называется следующее множество
\vspace{-4mm}
\[
C_{\gamma} = \{ \bx | f(\bx) \leq \gamma \}.
\]
\end{block}

\begin{block}{Квазивыпуклая функция}
Функция $f$ называется квазивыпуклой, если её область определения и множество подуровней для любых $\gamma$ выпуклые множества. 
\end{block}
\end{frame}

\begin{frame}{Критерии выпуклости}
\footnotesize
\vspace{-2mm}
\begin{block}{Дифференциальный критерий первого порядка}
Функция $f$ выпукла $\Leftrightarrow$ она определена на выпуклом множестве~$X$ и $\forall \bx, \by \in X \subset \bbR^n$ выполнено:
\vspace{-4mm}
\[
f(\by) \geq f(\bx) + \left( \nabla f(\bx) \right)^{\T} (\by - \bx)
\]
\end{block}

\begin{block}{Дифференциальный критерий второго порядка}
Непрерывная и дважды дифференцируемая функция $f$ выпукла $\Leftrightarrow$ она определена на выпуклом множестве $X$ и $\forall \bx \in \textbf{relint}(X) \subset \bbR^n$ выполнено:
\vspace{-2mm}
\[
\nabla^2 f(\bx) \succeq 0.
\]
\end{block}

\begin{block}{Связь с надграфиком}
Функция выпукла $\Leftrightarrow$ её надграфик выпуклое множество.
\end{block}

\begin{block}{Ограничение на прямую}
Функция $f: X \rightarrow \bbR$ выпукла тогда и только тогда, когда $X$ выпуклое множество и выпукла функция $g(t) = f(\bx + t\bv)$ на множестве $\{ t \mid \bx + t\bv \in X \}$ для всех $\bx \in \mathrm{dom}(f)$ и $\bv \in \mathbb{R}^n$.
\end{block}

\end{frame}

\begin{frame}{Критерии сильной выпуклости}

\begin{block}{Дифференциальный критерий первого порядка}
Функция $f$ сильно выпукла с константой $m$ $\Leftrightarrow$ она определена на выпуклом множестве $X$ и $\forall \bx, \by \in X \subset \bbR^n$ выполнена:
\vspace{-4mm}
\[
f(\by) \geq f(\bx) + \left( \nabla f(\bx) \right)^{\T} (\by - \bx) + \frac{m}{2}\| \by - \bx \|^2
\]
\end{block}

\begin{block}{Дифференциальный критерий второго порядка}
Непрерывная и дважды дифференцируемая функция $f$ сильно выпукла с константой $m$ $\Leftrightarrow$ она определена на выпуклом множестве $X$ и $\forall \bx \in  \bbR^n$ выполнено:
\vspace{-2mm}
\[
\nabla^2 f(\bx) \succeq m\bI.
\]
\end{block}
\end{frame}

\begin{frame}{Примеры}
\begin{enumerate}[<+->]
\item Квадратичная функция: $f(x) = \frac{1}{2}\bx^{\T}\bP\bx + \bq^{\T}\bx + r$, $\bx \in \bbR^n$, $\bP \in \bS^n$
\item Нормы в $\bbR^n$
\item $f(\bx) = \log{(e^{x_1} + \ldots + e^{x_n})}$, $\bx \in \bbR^n$~--- гладкое приближение максимума
\item Логарифм детерминанта: $f(\bX) = -\log{\det{\bX}}$, $\bX \in \bS^n_{++}$
\item Множество выпуклых функций~--- выпуклый конус
\item Поэлементный максимум выпуклых функций: $f(\bx) = \max\{f_1(\bx), f_2(\bx)\}$, dom $f$ = dom $f_1 \; \cap $ dom $f_2$
\item Расширение на бесконечное множество функций: если для $\by \in \calA$ функция $f(\bx, \by)$ выпуклая функция по $\bx$, тогда $\sup\limits_{\by \in \calA} f(\bx, \by) $ выпукла по $\bx$
\item Максимальное собственное значение: $f(\bX) = \lambda_{\max}(\bX)$ 

\end{enumerate}
\end{frame}

\begin{frame}{Неравенство Йенсена}
 
\begin{block}{Неравенство Йенсена}
Для выпуклой функции $f$ выполнено следующее неравенство:
\vspace{-4mm}
\[
f\left( \sum\limits_{i=1}^n \alpha_i \bx_i \right) \leq \sum\limits_{i=1}^n \alpha_i f(\bx_i),
\vspace{-4mm}
\] 
где $\alpha_i \geq 0$ и $\sum\limits_{i=1}^n \alpha_i = 1$.
\end{block}

или в бесконечномерном случае: $p(x) \geq 0$ и $\int\limits_X p(x) = 1$ 
\vspace{-4mm}
\[
f\left( \int\limits_X p(x)xdx \right) \leq \int\limits_X f(x)p(x)dx
\]
при условии, что интегралы существуют.

\end{frame}

\begin{frame}{Примеры}
\begin{enumerate}
\item Неравенство Гёльдера
\item Неравенство о среднем арифметическом и среднем геометрическом
\item $f(\bE(x)) \leq \bE(f(x))$
\item Выпуклость множества $\{ \bx | \prod\limits_{i=1}^n x_i \geq 1 \}$
\end{enumerate}
\end{frame}

\begin{frame}{Резюме}
\begin{itemize}
\item Выпуклая функция
\item Надграфик и множество подуровня функции
\item Критерии выпуклости функции
\item Неравенство Йенсена
\end{itemize}
\end{frame}
\end{document}
